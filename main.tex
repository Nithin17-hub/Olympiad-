\documentclass{article}
\usepackage{gvv}
\usepackage{amsmath}
\begin{document}
\begin{enumerate}
		\section*{Sixteenth International Olympiad, 1974}
		\subsection*{1974/1 \ Algebra}
\item Three players $A,B$ and $C$ play the following game. On each of three cards an integer is written. These three numbers $p, q, r$ satisfy$ 0 \textless p \textless q \textless r$. The three cards are shuffled and one is dealt to each player. Each then receives the number of counters indicated by the card he holds. Then the cards are shuffled again, the counters remain with the players. This process \brak{shuffling, dealing, giving out counters} takes place for at least two rounds. After the last round, $A$ has $20$ counters in all, $B$ has $10$ and $C$ has $9$. At the last round $B$ received $r$ counters. Who received $q$ counters on the first round?

	\subsection*{1974/2 \ Geometry}
\item In the triangle $ABC$, prove that there is a point $D$ on side $AB$ such that $CD$ is the geometric mean of $AD $and $DB$ if and only if
	\begin{align}\sin A \sin B \leq \sin 2 \frac{C}{2}.\end{align}
		\subsection*{1974/3 \ Number System}
	\item Prove that the number \begin{align} \sum_{k=0}^{n}\binom{2n+1}{2k+1}2^{3k}
	\end{align} is not divisible by $5$ for any integer
		$n \geq 0.2$
		\subsection*{1974/4 \ Geometry}
	\item Consider decompositions of an $8\times8$ chessboard into $p$ non-overlapping rectangles subject to the following conditions.\\ \brak{i} Each rectangle has as many white squares as black squares.\\ \brak{ii} If $ai$ is the number of white squares in the $i-th$ rectangle, then $a1 \textless a2 \textless \ldots \textless ap$. Find the maximum value of$ p$ for which such a decomposition is possible. For this value of $p$, determine all possible sequences $a1, a2, \ldots, ap$.
		\subsection*{1974/5 \ Number System}
	\item Determine all possible values of \begin{align} S=\frac{a}{a+b+d}+\frac{b}{a+b+c}+\frac{c}{b+c+d}+\frac{d}{a+c+d} \end{align} 
			Where $a,b,c,d$are arbitrary numbers.
		\subsection*{1974/6 \ Algebra}
		\item Let $P$ be a non-constant polynomial with integer coefficients. If $n\brak{P}$ is the number of distinct integers $k$ such that 
			\begin{align}\brak{P\brak{k}}2 = 1 \end{align}, prove that \begin{align}n\brak{P}-deg\brak{P} \geq 2\end{align},where $deg\brak{P}$ denotes the degree of the polynomial $P$.

\newpage
		\section*{Seventeenth International Olympiad, 1975}
		\subsection*{1975/1 \ Algebra}
	\item Let $x_{i}, y_{i} \brak{i=1, 2,\ldots, n}$ be real numbers such that $x_{1} \geq x_{2} \geq \ldots \geq x_{n}$ and $y_{1} \geq y_{2} \geq \ldots \geq y_{n}$.
		Prove that, if $z_{1}, z_{2},\ldots, z_{n}$ is any permutation of $y_{1}, y_{2},\ldots , y_{n}$, then
		\begin{align} \sum_{i=1}^{n}(x_{i}-y_{i})^{2}\leq\sum_{i=1}^{n}(x_{i}-z_{i})^{2}
		\end{align}
		\subsection*{1975/2 \ Number System}
	\item Let $a_{1}, a_{2}, a_{3}, \ldots  a_{n}$ infinite increasing sequence of positive integers. Prove that for every $p \geq 1$ there are infinitely many am which can be written in the form
		\begin{align}a_{m} = x_{ap} + y_{aq} \end{align}
			with $x, y$positive integers and $q \textgreater p$.

			\subsection*{1975/3 \ Geometry}
		\item On the sides of an arbitrary triangle $ABC$, triangles $ABR, BCP, CAQ $are constructed externally with \begin{align}\angle CBP = \angle CAQ = 45^o, \angle BCP = \angle ACQ = 30^, \angle ABR = \angle BAR = 15^o\end{align}. Prove that \begin{align}\angle QRP = 90^o and QR = RP\end{align}.
\subsection*{1975/4 \ Number System}
			\item When $4444^{4444}$ is written in decimal notation, the sum of its digits is $A$. Let $B$ be the sum of the digits of $A$. Find the sum of the digits of $B$. $\brak{A and B are written in decimal notation}$.
\subsection*{1975/5 \ Geometry}

			\item Determine, with proof, whether or not one can find $1975$ points on the circumference of a circle with unit radius such that the distance between any two of them is a rational number.
\subsection*{1975/6 \ Algebra}
			\item Find all polynomials $P$, in two variables, with the following properties.\\ \brak{i} for a positive integer $n$ and all real $t, x, y$
				\begin{align} P\brak{ta,ty}=t^n P\brak{x,y} \end{align}
					\brak{that is, P is homogeneous of degree n},\\ \brak{ii} for all real $a, b, c$ \begin{align}P\brak{b + c, a} +P\brak{c + a, b} +P\brak{a + b, c} =0\end{align},\\ \brak{iii} $P\brak{1, 0} = 1$.

\newpage
		\section*{Eighteenth International Olympiad, 1976}
		\subsection*{1976/1 \ Geometry}
	\item In a plane convex quadrilateral of area $32$, the sum of the lengths of two opposite sides and one diagonal is $16$. Determine all possible lengths of the other diagonal.
\subsection*{1976/2 \ Algebra}
	\item Let\begin{align} P1\brak{x} = x2 - 2 and Pj\brak{x} = P1\brak{Pj-1\brak{x}} for j = 2, 3, \ldots \end{align}. Show that, for any positive integer $n$, the roots of the equation $Pn\brak{x} = x$ are real and distinct.
\subsection*{1976/3 \ Geometry}
		\item A rectangular box can be filled completely with unit cubes. If one places as many cubes as possible, each with volume $2$, in the box, so that their edges are parallel to the edges of the box, one can fill exactly $40\%$ of the box. Determine the possible dimensions of all such boxes.
\subsection*{1976/4 \ Number systems}
		\item Determine, with proof, the largest number which is the product of positive integers whose sum is $1976$.
\subsection*{1976/5 \ Algebra}
		\item Consider the system of $p$ equations in $q = 2p$ unknowns $x_{1}, x_{2}, \ldots, x_{q} $.
			\begin{align}a_{11}x_{1} + a_{12}x_{2} +\ldots + a_{1q}x_{q} = 0 a_{21}x_{1} + a_{22}x_{2} +\ldots + a_{2q}x_{q}= 0 \ldots \\a_{p1}x_{1} + a_{p2}x_{2} + \ldots + a_{pq}x_{q} = 0\end{align} with every coefficient $a_{ij}$ member of the set$ \{-1, 0, 1\}$. Prove that the system has a solution $\brak{x_{1}, x_{2},\ldots , x_{q}}$ such that.\\ $\brak{a} a_{ll} x_{j} \brak{j = 1, 2, \ldots q}$ are integers,\\ $\brak{b}$ there is at least one value of $j$ for which $x_{j} \neq 0$,\begin{align} \brak{c} \mydet{xj} \leq q\brak{j = 1, 2,\ldots, q}\end{align}.
\subsection*{1976/6 \ Number Systems}
				\item A sequence $\{u_{n}\}$ is defined by\begin{align}u_{0} = 2, u_{1} = \frac{5}{2}, u_{n+1} = u_{n}\brak{u_{2}^{n-1} - 2} - u_{1}for n = 1, 2, \ldots \end{align} Prove that for positive integers n,\begin{align} \myvec{u_{n}} = 2^{\frac{\myvec{2n-\brak{-1}n}}{3}}\end{align}where $\myvec{x}$ denotes the greatest integer $\leq x$.
\end{enumerate}
\end{document}

